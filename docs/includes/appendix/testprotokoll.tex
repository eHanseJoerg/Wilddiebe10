\section{Testprotokoll}
\captionof{table}{Identifizierte Maßnahmen}
\label{tab:massnahmen}
\begin{longtable}{p{6.8cm}p{2.4cm}p{2.4cm}p{3cm}}
\toprule
Funktion & Status Nord & Status Süd & Bemerkungen \\
\midrule
\textbf{Korrektheit und Vollständigkeit der Ansible Konfiguration }
\begin{itemize}
\item
  Server komplett zurücksetzen und Ansible Konfiguration durchführen
\item
  läuft durch ohne Fehler
\end{itemize} & OK & OK & \\
\midrule
\textbf{Einbinden Netzlaufwerk}

\begin{itemize}
\item
  Windows 8.1+ Client
\item
  AD Anmeldung
\item
  Transport via AES
\end{itemize} & & & \\
\midrule
\textbf{Daten schreiben auf Freigabe}

\begin{itemize}
\item
  Linux (Samba 4.x) Client
\end{itemize}

\begin{itemize}
\item
  Transport via AES
\end{itemize} & & & \\
\midrule
\textbf{Daten lesen von Freigabe}

\begin{itemize}
\item
  Windows 8.1+ Client
\item
  Transport via AES
\end{itemize} & & & \\
\midrule
\textbf{Einbinden Netzlaufwerk "Jeder"}

\begin{itemize}
\item
  Linux (Samba 4.x) Client
\end{itemize} & & & \\
\midrule
\textbf{Schreiben auf Jeder Freigabe }

\begin{itemize}
\item
  Windows 8.1+ Client
\item
  AD Anmeldung
\item
  Transport via AES
\end{itemize} & & & \\
\midrule
\textbf{Quotas}

\begin{itemize}
\item
  Windows 8.1+ Client
\item
  AD Anmeldung
\item
  Schreiben von \textgreater{}100 MB
\end{itemize} & & & \\
\midrule
\textbf{Quotas Jeder }

\begin{itemize}
\item
  Linux (Samba 4.x) Client
\item
  Schreiben von \textgreater{}100 MB
\end{itemize} & & & \\
\midrule
\textbf{Virus schreiben}

\begin{itemize}
\item
  EICAR-Testdatei auf Freigabe schreiben
\item 
  Eintrag von ClamAV im lokalen Log: Eicar-Test-Signature
\item 
  Eintrag im rsyslog auf anderem Server
\item
  Zugriff wird verhindert
\end{itemize} & & & \\
\midrule
\textbf{Zugriff via ipv6 nicht möglich}

\begin{itemize}
\item
  Zugriff von interner IP auf SMB Freigabe per IPv6
\item
  Zugriff von externer IP auf SMB Freigabe per IPv6
\end{itemize} & & & \\
\midrule
\textbf{Automatisches Backup}

\begin{itemize}
\item
  anlegen von Referenzdatei auf persönlicher Freigabe
\item
  anlegen von Referenzdatei auf Jeder-Freigabe
\item
  Überprüfung der Backup-Datei auf dem jeweils anderen Server nach 24h
\end{itemize} & & & \\
\midrule
\textbf{Syslog}

\begin{itemize}
\item
  Anmeldung mit Admin-Account auf Server
\item
  Ausführen von SUDO-ping
\item
  Anmeldung und ping sind im lokalen Log vorhanden
\item
  Anmeldung und ping sind im remote-log vorhanden
\end{itemize} & & & \\
\midrule
\textbf{Gesicherter Login}

\begin{itemize}
\item
  Test-Login mit Admin-Account und falschem Passwort wird nach max. 10
  Versuchen abgelehnt
\item
  Zeitpunkt des letzten Login und Logout werden beim Anmelden mitgeteilt
\end{itemize} & & & \\

\bottomrule
\end{longtable}
