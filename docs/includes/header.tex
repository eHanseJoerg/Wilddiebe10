\documentclass[12pt]{article}
\usepackage[backend=biber, style=alphabetic, sorting=ynt]{biblatex}
\usepackage[utf8]{inputenc} % Datei ist in UTF-8 encoded 
\usepackage[T1]{fontenc}   % font encoding
\usepackage{courier} % courier font, usage: \texttt{This is Courier font.}
\usepackage[german]{babel} % deutsche Umlaute
\usepackage{csquotes}
\usepackage{epigraph}  % ein epigraph ist ein schönes Zitat am Anfang
\usepackage{graphicx} % Required to insert images
\usepackage{mathtools} % absmath with benefits
\usepackage{listings} % source code listings
\usepackage{authblk} % multiple authors
\usepackage{xcolor} % colored things

\usepackage{fancyhdr} % Required for custom headers
\usepackage{extramarks} % Required for headers and footers

\usepackage{booktabs} % neat table formatting
\usepackage{longtable} % tables over more than one page
\renewcommand{\arraystretch}{1.2} % mehr Zeilenabstand bei Tabellen
\setlength\heavyrulewidth{0.25ex} % thicker top and bottom lines in tables
\newcommand{\tabitem}{~~\llap{\textbullet}~~} % use items in tables with \tabitem XXX
\usepackage[labelfont=bf]{caption} % captions for tables without float
\captionsetup[table]{justification=raggedright,singlelinecheck=false} % tables have left aligned captions
\usepackage{xparse}

\usepackage[bookmarks=true]{hyperref}
\usepackage{glossaries}

\makeglossaries

\newsavebox{\fminipagebox}
\NewDocumentEnvironment{framedminipage}{m O{\fboxsep}}
 {\par\kern#2\noindent\begin{lrbox}{\fminipagebox}
  \begin{minipage}{#1}\ignorespaces}
 {\end{minipage}\end{lrbox}%
  \makebox[#1]{%
    \kern\dimexpr-\fboxsep-\fboxrule\relax
    \fbox{\usebox{\fminipagebox}}%
    \kern\dimexpr-\fboxsep-\fboxrule\relax
  }\par\kern#2
 }

\addbibresource{Quellen.bib}

% Standard-Schriftart
\renewcommand{\familydefault}{\sfdefault}
\usepackage{helvet}

% Margins
\topmargin=-0.45in
\evensidemargin=0in
\oddsidemargin=0in
\textwidth=6.5in
\textheight=9.0in
\headsep=0.25in
\linespread{1.1} % Line spacing
\setlength\parindent{0pt} % Removes all indentation from paragraphs

% Nummerierung
\setcounter{secnumdepth}{4} % Tiefe in der Überschriften-Nummerierung endet
\setcounter{tocdepth}{3} % Tiefe des Inhaltsverzeichnisses
